\chapter{The FAIR data principles}
Many funding agencies require researchers to arrange research data management solutions, and ask research to create data that can be reused in the future. At first, this might seem difficult because how do you create data that you, or someone else, can use in the in future. 
\whiteline
Here, the FAIR data principles come into play.The FAIR data principles are a framework to make data \underline{F}indable, \underline{A}ccessible, \underline{I}nteroperable, and \underline{R}eusable. They are designed to describe and guide data publishing with respect to support deposition, exploration, sharing, and reuse. Increasing the FAIRness of your data provides a range of benefits:
\begin{itemize}
\item Achieving maximum impact from research.
\item Increasing the visibility and citations of research.
\item Improving the reproducibility and reliability of research.
\item Attracting new partnerships.
\item Enabling new research questions to be answered.
\end{itemize}
\whiteline
There are different levels of making your data FAIR. By applying some of the principles, you can already increase the FAIRness of your data. 
\whiteline
\textbf{\underline{Findable:}} It is not a coincidence that FAIR data starts with findable, without the ability to find data we cannot access or reuse it. You can make your data findable by using: \\
\begin{itemize}
\item Data are described with rich metadata.
\item (Meta)data are assigned and unique and persistent identifier (for example a DOI).
\item Metadata clearly and explicitly include the identifier of the data they describe.
\item (Meta)data are registered or indexed in a searchable resource.
\end{itemize}
\textbf{\underline{Accessible:}} once data is findable, others need to know how the data can be accessed (including authentication and authorization). To make data accessible:\\
\begin{itemize}
\item (Meta)data are retrievable by their identifier via a standardized communications protocol.
\item (Meta)data are accessible, even when the data are no longer available. 
\end{itemize}
\whiteline
\textbf{\underline{Interoperable:}} data usually needs to be integrated with other data. In addition, data needs to be interoperable with applications or workflows for analysis, storage, and processing.\\
\begin{itemize}
\item (Meta)data use a formal, accessible, shared, and broadly applicable language for knowledge representation.
\item (Meta)data uses vocabularies that follow FAIR principles.
\item (Meta)data includes qualified references to other (meta)data.
\end{itemize}
\whiteline
\textbf{\underline{Reusable:}} the ultimate goal of FAIR data is to optimize the reuse of data:
\begin{itemize}
\item (Meta)data are richly described with a plurality of accurate and relevant attributes.
\end{itemize}
\whiteline
As we can see, most of the points addressed by the FAIR principles highlight the importance of describing data and metadata. Indirectly, they also describe how to implement the infrastructure component. For example,  the fourth point under Findable: �(Meta)data are registered or indexed in a searchable resource� refers to the three entities: Data, metadata, and infrastructure. It defines that both metadata and data are registered or indexed in a searchable resource (the infrastructure component). \\
The rich description of data, by adding metadata, is a requirement which seems logical. In scientific research, communication between researchers, funding agencies, and other partners is essential. 
\whiteline
So far, we have only discussed the basics of the FAIR principles. When we look at the four points discussed above, we can quickly see that \textit{Documentation} and \textit{Metadata} is vital to improve the FAIRness of your data. Therefore, let us have a look into (meta)data documentation.

