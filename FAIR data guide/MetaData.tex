\subsection{Metadata: data about data}
So far we have talked about data documentation without any structure. There is a second type of data documentation that is structured: metadata. This structured information has the potential for machine-to-machine interoperability. (this is one of the main differnces between normal documentation and metadata, normal documentation is often nog readable for machines).
\whiteline
Metadata is usally structured according to an international standard which include metadata descriptors. 
\whiteline
As with normal data documentation, there are also three categories of metadata:
\begin{itemize}
	\item Descriptive metadata ( for example: subject, data creator, title, funder)
	\item Administrative metadata (rights statement, version of data, timestamp of transaction, file formats)
	\item Structural metadata (variable list, taxonomy, database schema, data dictionary)
\end{itemize}
\whiteline
To not drown in all the options and to create an uniform metadata documentation style, it helps to pick an existing metadata schema. A metadata schema’s documentation will detail which elements are mandatory or optional, which elements are repeatable, relationship between elements, and guides the user on how to input and format values. Note, the presence of such a schema depends on your research field.
There are different standards for structured metadata:
\whiteline
\begin{itemize}
	\item Dublin core
	\item Data documentation initiative (DDI)
	\item Statistical Data and Metadata eXchange (SDMX)
\end{itemize}
\whiteline
These standards provide guidelines for documentation your data. However, these standards are quite general and it might be better to pick a field-specific metadata scheme (when available).


