
\section{Metadata: data about data}
Metadata, in the context of file and object storage, plays a crucial role in providing structure in data management. 
Metadata is information about data to identify the properties of a file or object of any format. It can be seen as a data report: 
\begin{itemize}
\item \textit{Who} created the data?
\item \textit{What} is the content of the data?
\item \textit{When} were the data created?
\item \textit{Where} is the data generated?
\item \textit{How} are the data generated?
\item \textit{Why} is the data generated?
\end{itemize}
You might think \textit{"The answers to these questions can be found in the publication"}. Although the experimental section in a paper often contains most of the information, it is summarized and often technical details are omitted. 
\whiteline
Ideally you should prepare metadata at study level and at data level (for each data set within a study). This metadata file should contain:
\begin{itemize}
\item The aim(s) of the project.
\item The method(s) used to collect the data.
\item The content(s) of your data.
\item The folder structure and file naming conventions.
\item The data processing technique(s) used.
\item The modification made to the initial data throughout the project.
\item Data validation and other quality assurance processes.
\item Roles and responsibilities within a project.
\item Details in identifiers, licensing, and sensitive information.
\end{itemize}
Metadata, both on data and project level, can be further classified into three classes:
\begin{enumerate}
\item Descriptive metadata.
\item Structural metadata.
\item Administrative metadata.
\end{enumerate}
Descriptive metadata provides basic facts about an object, whereas structural metadata describes the structure of an object including its components and how they relate. Administrative metadata provides information about preservation and rights, creation date and file integrity checks.
\section{How to start and add documentation to your data}
Data documentation has to start at an early phase in the research life cycle. One of the most informative documents about a data set is a lab journal. As the primary working document, it contains all hypotheses, calculations, data, experimental setups, and early analysis are first recorded in lab notebooks before being transferred elsewhere.
\whiteline
\input{}