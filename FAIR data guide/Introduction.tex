\chapter{General Introduction}
As a researcher, you will spend a lot of time performing experiments to get data, and analyze the results to answer important questions. However, you will probably not work on only one data set but will obtain many throughout your career. You might know everything about your data set at this moment, but would you remember all the details in 5 years? \par 

Research data management is a term that described the organization, storage, preservation, and sharing of data collected and used in a research project. It involves the management of data during the lifetime of a project but also involves decisions about how data will be preserved and shared after the project is completed. 
Good management helps to prevent errors and increases the quality of your analysis. Data management saves time and resources in the long run, Furthermore, well-managed and accessible data allows others to validate and replicate your findings. Finally, by sharing data, it can lead to valuable discoveries by others outside of the original research team.
Good data management is not easy and it can be a challenge to start, especially if your working in an ongoing project.  \par

In this document, we will go over the various topics that cover the whole research data life cycle. We will introduce the FAIR data principles, the importance of data documentation, data storage, and the best practices for sharing research software. 
