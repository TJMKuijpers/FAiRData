\sebsection{Electronic lab journal}
An Electronic Lab Journal is a digital version that replicates an interface much like a page in your paper lab notebook. In your ELN you can enter protocols, observations, other data just like you would do on paper. ELNs offer several advantages over traditional paper notebooks; they facilitate good data management practices, providing for data security, auditing, and collaboration.
\whiteline
However, you should keep in mind that not every ELN is the same, Therefore it is important to choose an ELN that does meet your needs. It takes time and thought to set up an ELN. Your lab will need to come to a decision on how your data should be organized, and shared. 
\whiteline
There are various options for an ELN, and you have to decide which platform works best for you:
\begin{itemize}
\item Microsoft OneNote
\item Evernote
\item LabArchives
\item RSpace
\item eLabNext
\item Open Science Framework
\end{itemize}
\newline
Testing different options can be a good way to select the right ELN for you. In the end, it all depends on your needs, including:
\begin{itemize}
\item The type of specific features you need.
\item Your lab's established practices and preferences/
\item Your institution's ELN policies.
\item Your budget
\end{itemize}
The following Nature publication \textit{"How to pick an electronic laboratory notebook" (DOI:10.1038/d41586-018-05895-3)} might be a helpful guide to select the right ELN for your lab.