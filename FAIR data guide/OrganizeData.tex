\chapter{Organize your data in a structured way}
At the start of your project, you want to create a clear folder structure and file name convention. When you define a project folder at the start of a project, you can easily organize all the data you gather during the project. Once the project is finished, you should have one folder containing all the raw data, analyzed data, results and documentation in one place. Furthermore, folder structures enable research processes to be more transparent.

The hierarchy of  your folders should be consistent and logical. Go from a general, high-level folder to more specific lower-level folders. Your structure should not be too deep or too shallow. 

One option is to organize folder levels based on research activity, data type, and kind of content. 

There are a couple of things to avoid:
\begin{itemize}
	\item Do not use a generic 'current stuff' folder
	\item Do not create researcher-specific folders within a project (folders are about data and not reseachers)
	\item Make sure you do not have overlapping categories or folder redundancy
	\item Do not create copies of files in different folders (use shortcuts if you need this)
\end{itemize}

\subsection{File naming convention}
Once you have created your project folder structure, you can start to fill the folders with data files. However, it is important to name your files in a logical way. It does not make sense to have an organized folder structure and name your files 'output1.txt', output2.txt, and so on. Let's distinguish between the naming of files containing experiment data and all other files (both research-related and non-research-related).
\whiteline
Applying a consistent and descriptive file naming convention (i.e. a systematic file naming method) helps to:
\begin{itemize}
	\item Identify the content of a file without opening it
	\item Easily and quickly locate, retrieve, and filter data(files), even if they have changed folders
	\item Easily sort and browse through your files
	\item Identify missing data(files)
\end{itemize}
\whiteline
There are a couple of things you can consider while creating a naming convention. 
\begin{enumerate}
	\item \textbf{How to do want to sort your files}? This can be by:
		\begin{enumerate}
			\item Date: use YYYYMMDD or YYYY-MM-DD, which ensures sorting in chronological order.
			\item File version: use leading zeros, v01 or v001 instead of v1 to ensure sequential sorting
			\item Other important parameters, that are most important to you
			\end{enumerate}
		\item \textbf{Avoid long file names} file names should not exceed 30 characters. If you need to include many parameters in your file name, you could abbreviate them. When you abbreviate words, it is important to document this in a readme.txt 
	\item \textbf{Avoid special characters and spaces} when separating the different elements of your file name, do not use spaces or special characters because some software programs don't recognize file names with these characters. You can use underscores or capitalize the first letter of the new word 
\end{enumerate}
